\documentclass{muratcan_cv}

\setname{Breno}{Marques Azevedo}
\setaddress{Rio de Janeiro/Brasil}
\setbirth{Brasileiro | Medina - MG}
\setmobile{31 99884 8546}
\setmail{eborn.azevedo@gmail.com}
\setposition{estagiário, intern, intership} %ignored for now
\setlinkedinaccount{https://www.linkedin.com/in/brenomazevedo/} %you can play with color of the template (red is also nice..)
\setgithubaccount{https://github.com/Breno-Azevedo} %you can play with color of the template (red is also nice..)
\setthemecolor{red} %you can play with color of the template (red is also nice..)

\begin{document}
%Set variables
%You can add sections, texts, explanations just by copying the style below. Replace the dummy texts "\lipsum[1][x-x]\par" with actual texts.
%Create header
\headerview
\vspace{1ex}
%Sections
%
% Summary
%\addblocktext{Resumo}{%
%%\lipsum[1][1-12]\ %replace this part with actual text
%\noindent 
%Estudante de Bacharelado em Ciência de Dados e Inteligência Artificial pela Fundação Getulio Vargas (FGV-RJ).
%
%Técnico em Informática formado pelo Instituto Federal de Educação, Ciência e Tecnologia do Norte de Minas Gerais (IFNMG - Campus Araçuaí).
%}

% Goals
%
\vspace{3mm}
\addblocktext{Objetivo}{%
%Em busca de uma primeira experiência no mercado como estagiário em projetos multidisciplinares que envolvam ciência de dados.
%Em busca de uma primeira experiência no mercado de trabalho como estagiário em projetos que envolvam business intelligence e banco de dados.
Em busca de experiência inaugural no mercado de trabalho como Cientista de Dados Júnior.
}

%
% Resumo
\addblocktext{Sumário}{
Profundamente intrigado pela aplicação de estatística na área da saúde e interessado na utilização mais ampla de modelos de linguagem em vários domínios.
}


%
%Education
\section{Formação Acadêmica} 
    \datedexperience{Fundação Getulio Vargas}{2021-Atual} 
    \explanation{Graduação em Ciência de Dados e Inteligência Artificial} 
    \explanationdetail{\coloredbullet\ %
     Previsão de formatura: Dezembro de 2024.
    }
    \datedexperience{Instituto Federal do Norte de Minas Gerais}{2017-2019} 
    \explanation{Curso Técnico em Informática} 
    \explanationdetail{\coloredbullet\ % 
     Formação com foco em Desenvolvimento Web e programação.
    }
%
% Experience
\section{Experiência}
    \datedexperience{Estagiário em Ciência de Dados - Perdigão \& Partners Business Solutions}{Nov/2023-Atual}
    \explanation{%
    Responsabilidades realizadas:
    }
    \explanationdetail{\coloredbullet\ %
        Atuando em um serviço de consultoria especializado em análise e soluções financeiras para hospitais e clínicas oncológicas na aquisição de materiais e medicamentos.
    }
    \explanationdetail{\coloredbullet\ %
        Estruturação e análise de dados;
    }
    \explanationdetail{\coloredbullet\ %
        Estruturação de bando de dados;
    }
    \explanationdetail{\coloredbullet\ %
        Criação de ferramentas de automação para simplificar a inserção de dados, otimizando a eficiência operacional.
    }
    \explanationdetail{\coloredbullet\ %
        Desenvolvimento de algoritmos e modelos prospectivos para prever tendências financeiras e auxiliar na tomada de decisões estratégicas.
    }
    \explanationdetail{\coloredbullet\ %
        Elaboração de dashboards interativos e intuitivos para visualização de dados, proporcionando insights valiosos para os stakeholders.
    }
%
% Extracurricular Experience
\section{Atividades Extracurriculares}
    %
    \datedexperience{InfoDengue - Fundação Oswaldo Cruz (Fiocruz)}{Jul/2020-Out/2020} 
    \explanation{%
    Auxílio no Desenvolvimento e Manutenção de um Website
    }
    \explanationdetail{\coloredbullet\ %
    Sistema de alerta para arboviroses e análise integrada de dados epidemiológicos e climáticos.
    }
    \explanationdetail{\coloredbullet\ %
    A atividade envolvia a manutenção e desenvolvimento do site Info Dengue, desenvolvido usando o framework Django do Python.
    }
    %
    \datedexperience{Open Labeller for Iterative Machine Learning - Organização Mundial da Saúde}{Mar/2023-Jul/2023}
    \explanation{%
    %Auxílio no Desenvolvimento e Testagem de um rotulador de pacientes com base em prontuários.
    Auxílio no desenvolvimento de uma plataforma que identifica os efeitos tardios do Corona vírus.
    }
    \explanationdetail{\coloredbullet\ %
    %Rotulador de pacientes com base em textos de prontuários, desenvolvido em projeto sob tutela da OMS, com o objetivo de apontar os efeitos tardios do Covid-19.
    O objetivo era fazer uso de modelos de Machine Learning para identificar os sintomas e efeitos tardios em pacientes que tiveram contato com o Corona vírus.
    }
    \explanationdetail{\coloredbullet\ %
    %As atividades envolviam a coleta, limpeza e rotulação de dados médicos relacionados a pacientes que tiveram contato com Covid-19, com o objetivo de auxiliar no treinamento de modelos de Machine Learning.
    As atividades envolviam coleta, limpeza e rotulação de dados extraídos de prontuários, com o objetivo de auxiliar no treinamento de modelos de Machine Learning.
    }
    %
    \datedexperience{Bahia Asset Management|Field Project}{Ago/2023-Dez/2023}
    \explanation{O Field Project teve como objetivo estimar a produção de milho dos EUA utilizando dados climáticos.}
    \explanationdetail{\coloredbullet\ %
        Interatividade com um grande volume de dados climáticos;
    }
    \explanationdetail{\coloredbullet\ %
        Realização de análise exploratória dos dados para embasar decisões informadas;
    }
    \explanationdetail{\coloredbullet\ %
        Implementação de metodologias específicas e desenvolvimento de modelos para cada estado;
    }
    %\datedexperience{Velocity Inc.}{2015-Summer / Turkey} 
    %\explanation{Intern as Developer/Tester} 
    %
    %\datedexperience{Akbank}{2018-2019 / Turkey} 
    %\explanation{Ios Developer} 
    %\explanationdetail{\coloredbullet\ %
    % \lipsum[1][1-2]\par %replace this part with actual text
    % }
    %
    %\datedexperience{Mobile-Software AG}{2019-2020 / Germany} 
    %\explanation{Ios Developer} 
    %\explanationdetail{\coloredbullet\ %
    % \lipsum[1][4-5]\par %replace this part with actual text
    % }
    %
    %\datedexperience{BMW Autonomous Driving Campus}{2020-Now / Germany}
    %\explanation{Working Student} 
    %\explanationdetail{\coloredbullet\ %
    % \lipsum[1][3-4]\par %replace this part with actual text
    % }
%
% Skills
\vspace{3mm}
\section{Habilidades}
    %
    \newcommand{\skillone}{\createskill{Linguagens de Programação}{\textbf{\emph{Familiar:}} \ \  C++ \cpshalf Python \cpshalf SQL}}
    %
    \newcommand{\skilltwo}{\createskill{Desenvolvimento de Software}{GIT \cpshalf Docker}}
    %
    \newcommand{\skillthree}{\createskill{Frameworks \ \& \ Libraries}{Jupyter \cpshalf Matplotplib \cpshalf Numpy \cpshalf Pandas \cpshalf Scikit-learn \cpshalf Nltk \cpshalf Django \cpshalf Seaborn \cpshalf PySpark}}
    %
    %\newcommand{\skillfour}{\createskill{iOS Programming}{RxSwift \cpshalf PromiseKit \cpshalf CocoaPods \cpshalf Autolayout/DSLs}}
    %
    \newcommand{\skillfour}{\createskill{Idiomas}{\textbf{\emph{Fluente:}} \ \ Inglês \ \ }}
    %
    \newcommand{\skillfive}{\createskill{Pacote Office}
    {\textbf{\emph{Intermediário:}} \ \ Word \ \ Excel}}
    %
    \createskills{\skillone, \skilltwo, \skillthree, \skillfour, \skillfive}
%
% Experience
\vspace{2mm}
\section{Reconhecimentos e Certificados}
    \newcommand{\extraone}{%
    Bolsista pelo Centro para o Desenvolvimento da Matemática e Ciências (CDMC), programa que seleciona jovens talentos com capacidade especial para o aprendizado da Matemática.
    }
    \newcommand{\extrafour}{%
    Certificado de inglês intermediário, referente ao nível B2 do Quadro Europeu Comum de Referências.
    }
    \newcommand{\extratwo}{%
    Medalhista na Olimpíada Brasileira de Matemática das Escolas Públicas (OBMEP) de 2016.
    }
    %
    \newcommand{\extrathree}{%
    Interesse em Machine Learning e Redes Neurais.
    }
    %
    \newcommand{\listofextras}{\extraone \vspace{0.5mm}, \extrafour \vspace{0.5mm}, \extratwo}
    %
    \createbullets{\listofextras}
%
%Footnote
%\createfootnote
\end{document}
